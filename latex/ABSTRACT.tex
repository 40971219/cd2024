\renewcommand{\baselinestretch}{1.5} %設定行距
\pagenumbering{roman} %設定頁數為羅馬數字
\setcounter{page}{2}
\sectionef
\addcontentsline{toc}{chapter}{ABSTRACT} %將摘要加入目錄
\begin{center}
\LARGE\textbf{ABSTRACT}\\
\LARGE\textbf{摘要}\\
\end{center}
\fontsize{14}{18}\sectionef \textbf
 {ANALYSIS  OF  THE ODOO  SOFTWARE  CAPABILITIES  REGARDING  
 PRODUCT  LIFECYCLE  MANAGEMENT,  MANUFACTURING  EXECUTION 
 SYSTEMS  AND  THEIR  INTEGRATION }。\\[2pt]
\fontsize{16}{18}\sectionef \textbf
  {(ODOO 軟體功能分析產品生命週期管理、製造執行系統及其集成)}。\\[15pt]

\fontsize{14pt}{2.5pt}\sectionef 
{ The second half of the 20th century had been marked for the advancements of computer 
technology in all aspects of production.}。\\[1pt]

\fontsize{14pt}{5pt}\sectionef
 {20世紀下半葉以電腦的進步為標誌生產各個環節的技術}\\[15pt]

\fontsize{14pt}{2.5pt}\sectionef 
{ The key feature of that statement is the undeniable truth that alongside the increased 
complexity allowed by computing power comes an ever increasing production of 
overwhelming amounts of information. }。\\[1pt]

\fontsize{14pt}{5pt}\sectionef
 {該聲明的關鍵特徵是不可否認的事實,即隨著增加的計算能力所允許的複雜性,帶來了不斷增加的產量與海量資訊。}\\[15pt]

\fontsize{14pt}{2.5pt}\sectionef 
{From separate perspectives of the industrial landscape, several systems were brewed by 
that sheer necessity for organization, automation and waste reduction focusing on that pool 
of useful data. }。\\[1pt]

\fontsize{14pt}{5pt}\sectionef
 {從工業景觀的不同角度來看,出於組織、自動化和減少浪費的絕對必要性,一些系統誕生了,這些系統專注於有用資料池。}\\[15pt]

\fontsize{14pt}{2.5pt}\sectionef 
{ERP (from a managerial perspective), MES (from a production perspective) and more 
recently PLM (from a strategic development/redevelopment perspective) emerged as 
information solutions tackling this problem from different angles. These solutions, however 
effective, are always plagued by the fundamental incompatibility between the tools that 
implement those systems.}。\\[1pt]

\fontsize{14pt}{5pt}\sectionef
 {ERP(從管理角度)、MES(從生產角度)以及最近的 PLM(從策略開發/再開發角度)作為
資訊解決方案從不同角度解決這個問題。 這些解決方案無論多麼有效,總是受到實現這些系統的工具之間根本不相容的困擾。}\\[15pt]

\fontsize{14pt}{2.5pt}\sectionef 
{This paper objectives revolve around analyzing the integration PLM and MES systems 
from a theoretical perspective and comment on the use of the Odoo software tool to 
implement said integration.}。\\[1pt]

\fontsize{14pt}{5pt}\sectionef
 {本文的目標是從理論角度分析 PLM 和 MES 系統的集成,並對使用 Odoo 軟體工具實現所述集成進行評論。}\\[15pt]

\fontsize{14pt}{2.5pt}\sectionef 
{The Odoo software was described in detail (regarding its use for manufacturing 
envirorment) icluding how it implements PLM and MES. Then, the software was subjected 
to the simulation of a fictional firm devised in the molds of Industry 4.0. This company was
a fictional recently founded small case manufacturing company that uses plastic injection 
molding as their primary mean of production and uses additive manufacturing and fast 
prototyping as part of their business strategy.}。\\[1pt]

\fontsize{14pt}{5pt}\sectionef
 {詳細描述了 Odoo 軟體(關於其在製造環境中的使用),包括它如何實施 PLM 和 MES。 然後,該軟體對一家按照工業 4.0 模式設計的虛構公司進行模擬。 該公司是一家虛構的最近成立的小型箱體製造公司,使用塑膠注塑作為主要生產手段,並使用積層製造和快速原型製作作為其業務策略的一部分。}\\[15pt]

\fontsize{14pt}{2.5pt}\sectionef 
{Keywords: Product Life-Cycle Management, Product Life-Cycle Management, Odoo}。\\[1pt]

\fontsize{14pt}{5pt}\sectionef
 {關鍵字:產品生命週期管理、產品生命週期管理、Odoo}\\[15pt]
